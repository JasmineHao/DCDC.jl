\section{The baseline model}

The utility function is defined as:
\[ u(c_{h,t},l_{h,t}) = \frac{M^{1-\gamma}_{h,t}}{1 - \gamma} \exp{( \pi z_{h,t} + \zeta_{h,t} )} .\]
The preference aggregator (why aggregator) for hours of leisure and consumption is
\[ M_{h,t}(c_{h,t},l_{h,t};z_{h,t},\chi_{h,t}) = \left( \frac{(c_{h,t}^{1 - \phi} - 1)}{1 - \phi} + (\alpha_{h,t}(z_{h,t},\chi_{h,t})) \frac{(l_{h,t}^{1 - \theta} -1)}{ 1 - \theta}\right) \]
$\alpha_{h,t}= \exp{(\psi_0 + \psi_z z_{h,t} + \chi_{h,t})}$

Suppose for now there is no random discrete choice, the inter-temporal budget constraint:
\[ A_{h,t+1} = (1 + r_{t+1}) \left( A_{h,t} + (w_{h,t}^f(H-l_{h,t}) + y_{h,t}^m - c_{h,t} \right)\]
$A_{h,t}$ is the beginning asset holding, $r_t$ is the risk-free interest rate, $F$ is the fixed cost of work, dependent on the age of the youngest child, $a_{h,t}$. Female wages are given by $w_{h,t}^f$ and husband earning $y_{h,t}^m$.

The maximization is done with respect to the following payoffs:
\[\begin{split} \max_{c_t,l_t} & u(c_t,l_t) + \beta V(A_{t+1}) \\
\text{subject to } & A_{t+1} = (1 + r_{t+1}) \left( A_{t} + w_{h,t}^f(H-l_{t}) + y_{h,t}^m - c_{t} \right)
\end{split}\]
The first order condition with respect to $c_t, l_t and A_t$ are \[ \begin{split}
  u_c(c_t,l_t) + \beta (1+r_{t+1}) V'(A_{t+1}) = 0 \\
  u_l(c_t,l_t) + \beta w_{h,t} (1+r_{t+1}) V'(A_{t+1}) = 0 \\
  V'(A_t) = \beta (1+r_{t+1}) V'(A_{t+1}) \\
\end{split} \]
From the model we can have two conditions
\[ \begin{split}
  \textbf{MRS condition:}& \quad w_{ht} u_{c,ht} = l_{c,ht}\\
  \textbf{Euler Equation:}& \quad u_{c,ht} = \beta u_{c,ht+1}, \quad \text{in addition:}\quad u_{l,ht} = u_{l,ht+1} \frac{w_{h,t}}{w_{h,t+1}}
\end{split} \]

\textbf{The regression from MRS}
Therefore from the MRS condition, we have that
\[\begin{split}
   w_{h,t}c_{h,t}^{-\phi} &= \alpha_{h,t}(z_{h,t},\chi_{h,t})) l_{h,t}^{-\theta}  \\
   \log(w_{h,t})& + \phi \log(c_{h,t})  = \psi_0 + \psi_1 z_{h,t} + \xi_{h,t} - \theta \log(l_{h,t})
\end{split} \]
\textbf{The parameters can be identified from MRS:}
\begin{itemize}
  \item We observe the wages, consumptions, leisure time and the exogenous variables: $\{w,c,l,z\}$
  \item From the reduced form regression, we have the preference shocks identified ($\psi_0,\psi_1$) with the assumption that $\xi_{h,t}$ is normal.
  \item We also identify the structural preference parameter $\phi$ and $\theta$.
  \item After we obtain $\hat{\phi},\hat{\theta},\hat{\psi}_0,\hat{\psi}_1$ we cam back out $\hat{M}(c_{h,t},l_{h,t})$
  \item[Question:] {\color{red} Then $\log M(c_{h,t}, l_{h,t}) - \log \hat{M}(c_{h,t},l_{h,t})$ follows the assumed distribution of $\xi_{h,t}$? Is there potential endogeneity problem? ($\xi$ correlated with wage? other explainatory variables?)}
\end{itemize}
\textbf{The Euler equation and what can be identified from it?}
The Euler equation condition is that
\[ \begin{split}
  & \exp{( \pi z_{h,t} + \zeta_{h,t} )} M(c_{h,t},l_{h,t})^{-\gamma} c_{h,t}^{-\phi} \\
  & = \beta (1+r_{t+1}) \exp{( \pi z_{h,t+1} + \zeta_{h,t+1} )} M(c_{h,t+1},l_{h,t+1})^{-\gamma} c_{h,t+1}^{-\phi} \\
  M(c_{h,t},l_{h,t}) & = \left( \frac{(c_{h,t}^{1 - \phi} - 1)}{1 - \phi} + (\alpha_{h,t}(z_{h,t},\chi_{h,t})) \frac{(l_{h,t}^{1 - \theta} -1)}{ 1 - \theta}\right)
\end{split} \]
Log-linearize and plug in the estimated utility to have that
\[ \begin{split}
  \pi z_{h,t} & + \zeta_{h,t} - \phi c_{h,t} - \gamma \log(\hat{M}(c_{h,t},l_{h,t})) \\ & = \log(\beta) + \log(1+r_{t+1}) + \pi z_{h,t+1} + \zeta_{h,t+1} - \gamma \log(\hat{M}(c_{h,t+1},l_{h,t+1})) - \phi c_{h,t+1} \\
  \text{Then we have}& \quad  \pi \Delta z_{h,t} + \Delta \zeta_{h,t} - \hat{\phi} \Delta c_{h,t} - \gamma \Delta \log \hat{M}(c_{h,t},l_{h,t}) + \beta + \log(1+r_t)= 0
\end{split} \]
What parameters we can estimate from the Euler equation
\begin{itemize}
  \item It feels wierd that no transition of $z_{h,t}$ is assumed? But may be we don't need this assumption.
  \item We can estimate the parameters $\pi,\gamma,\beta$ from the data?
  \item {\color{red} The $r_{t}$ is observed? The paper uses group aggregate labour supply or taxation as an IV for this step's identification}
  \item [Comment:] not sure what harm this would bring to estimation...But is this neccesary if we want to aggregate the $M_{h,t}$?
  \item [Question:] Page 16 of paper \cite{Attanasio2015AggregatingSupply}, lower part $M_{h,t}$ is non-linear function and unobserved parameters, so cannot be aggregated??
\end{itemize}

\section{Section with discrete choice}
The additional parameters are $\xi$ and $F(a_{h,t})$ where $F(a_{h,t})$
If there is a discrete choice to participate in labour market, then the utility function is defined as:
\[ u(c_{h,t},l_{h,t},P_{h,t}) = \frac{M^{1-\gamma}_{h,t}}{1 - \gamma} \exp{(\xi P_{h,t} + \pi z_{h,t} + \zeta_{h,t} )} .\]
\[ M_{h,t}(c_{h,t},l_{h,t};z_{h,t},\chi_{h,t}) = \left( \frac{(c_{h,t}^{1 - \phi} - 1)}{1 - \phi} + (\alpha_{h,t}(z_{h,t},\chi_{h,t})) \frac{(l_{h,t}^{1 - \theta} -1)}{ 1 - \theta}\right) \]
$\alpha_{h,t}= \exp{(\psi_0 + \psi_z z_{h,t} + \chi_{h,t})}$
The intertemporal budget constraint is defined as:
\[ A_{h,t+1} = (1 + r_{t+1}) \left( A_{h,t} + \left(w_{h,t}^f(H-l_{h,t}) - F(a_{h,t}) \right)P_{h,t} + y_{h,t}^m - c_{h,t} \right)\]
\textbf{The parameters that we can get from MRS and Euler Equation}
\begin{itemize}
  \item Similarly, estimate the parameters within $M(c_{h,t},l_{h,t})$ function and obtain $\hat{M}(c_{h,t},l_{h,t})$.
  \item The Euler equation is slightly different: \[ \xi \Delta P_{h,t}  \pi \Delta z_{h,t} + \Delta \zeta_{h,t} - \hat{\phi} \Delta c_{h,t} - \gamma \Delta \log \hat{M}(c_{h,t},l_{h,t}) + \beta + \log(1+r_t)= 0\]
  \item $\xi$ is the entry cost?
\end{itemize}
\tbf{Random shocks: what's confusing to me.}
\begin{itemize}
  \item No entry cost of labour market.
  \item The shock $\zeta$ induce increase of current period consumption/labour.
  \item Are $\zeta$ and $\chi$ correlated?
\end{itemize}

\section{My estimation strategy?}
I would propose a 3-step estimation strategy.
In the first step, in addition to estimating $MRS$, I will need to estimate the transition of $z_{it}$ as well.(So that the agents have a prior about the future events.)
From the first step, we obtain the estimator of $\log \hat{M}(c,h)$, and we also have the estimator of $w_{h,t}$.

In the second step, estimate the entry decision of the labour market.

In the third step, estimate the forward looking continuous choice.
