\section{The baseline model}

The utility function is defined as:
\[ u(c_{h,t},l_{h,t}) = \frac{M^{1-\gamma}_{h,t}}{1 - \gamma} \exp{( \pi z_{h,t} + \zeta_{h,t} )} .\]
The preference aggregator (why aggregator) for hours of leisure and consumption is
\[ M_{h,t}(c_{h,t},l_{h,t};z_{h,t},\chi_{h,t}) = \left( \frac{(c_{h,t}^{1 - \phi} - 1)}{1 - \phi} + (\alpha_{h,t}(z_{h,t},\chi_{h,t})) \frac{(l_{h,t}^{1 - \theta} -1)}{ 1 - \theta}\right) \]
$\alpha_{h,t}= \exp{(\psi_0 + \psi_z z_{h,t} + \chi_{h,t})}$

Suppose for now there is no random discrete choice, the inter-temporal budget constraint:
\[ A_{h,t+1} = (1 + r_{t+1}) \left( A_{h,t} + \left(w_{h,t}^f(H-l_{h,t}) + y_{h,t}^m - c_{h,t} \right)\]
$A_{h,t}$ is the beginning asset holding, $r_t$ is the risk-free interest rate, $F$ is the fixed cost of work, dependent on the age of the youngest child, $a_{h,t}$. Female wages are given by $w_{h,t}^f$ and husband earning $y_{h,t}^m$.

The maximization is done with respect to the following payoffs:
\[\begin{split} \max_{c_t,l_t} & u(c_t,l_t) + \beta V(A_{t+1}) \\
\text{subject to } & A_{t+1} = (1 + r_{t+1}) \left( A_{t} + \left(w_{h,t}^f(H-l_{t}) + y_{h,t}^m - c_{t} \right)
\end{split}\]
The first order condition with respect to $c_t, l_t and A_t$ are \[ \begin{split}
  u_c(c_t,l_t) + \beta (1+r_{t+1}) V'(A_{t+1}) = 0 \\
  u_l(c_t,l_t) + \beta w_{h,t} (1+r_{t+1}) V'(A_{t+1}) = 0 \\
  V'(A_t) = \beta (1+r_{t+1}) V'(A_{t+1}) \\
\end{split} \]
From the model we can have two conditions
\[ \begin{split}
  \textbf{MRS condition:}& \quad w_{ht} u_{c,ht} = l_{c,ht}\\
  \textbf{Euler Equation:}& \quad u_{c,ht} = \beta u_{c,ht+1}\\
  \text{in addition:}& \quad u_{l,ht} = u_{l,ht+1} \frac{w_{h,t}}{w_{h,t+1}}
\end{split} \]
\[ \begin{split}
u_c(c,l) & = \exp{( \pi z_{h,t} + \zeta_{h,t} )} \left( \frac{(c_{h,t}^{1 - \phi} - 1)}{1 - \phi} + (\alpha_{h,t}(z_{h,t},\chi_{h,t})) \frac{(l_{h,t}^{1 - \theta} -1)}{ 1 - \theta}\right)^{-\gamma} c_{h,t}^{-\phi} \\ \text{ and }u_l(c,l) & = \exp{( \pi z_{h,t} + \zeta_{h,t} )} \left( \frac{(c_{h,t}^{1 - \phi} - 1)}{1 - \phi} + (\alpha_{h,t}(z_{h,t},\chi_{h,t})) \frac{(l_{h,t}^{1 - \theta} -1)}{ 1 - \theta}\right)^{-\gamma} \alpha_{h,t}(z_{h,t},\chi_{h,t})) l_{h,t}^{-\theta}  \end{split}\]
 \textbf{The parameters can be identified from MRS:}
 \begin{itemize}
   \item 
 \end{itemize}
Therefore from the MRS condition, we have that
\[ w_{h,t}c_{h,t}^{-\phi} =\alpha_{h,t}(z_{h,t},\chi_{h,t})) l_{h,t}^{-\theta}  \]
The Euler equation condition is that
\[ \exp{( \pi z_{h,t} + \zeta_{h,t} )} \left( \frac{(c_{h,t}^{1 - \phi} - 1)}{1 - \phi} + (\alpha_{h,t}(z_{h,t},\chi_{h,t})) \frac{(l_{h,t}^{1 - \theta} -1)}{ 1 - \theta}\right)^{-\gamma} c_{h,t}^{-\phi} = \beta \]
\tbf{Random shocks}
\begin{itemize}
  \item No entry cost of labour market.
  \item The shock $\zeta$ induce increase of current period consumption/labour.
  \item Are $\zeta$ and $\chi$ correlated?
\end{itemize}

\section{Section with discrete choice}
The additional parameters are $\xi$ and $F(a_{h,t})$ where $F(a_{h,t})$
If there is a discrete choice to participate in labour market, then the utility function is defined as:
\[ u(c_{h,t},l_{h,t},P_{h,t}) = \frac{M^{1-\gamma}_{h,t}}{1 - \gamma} \exp{(\xi P_{h,t} + \pi z_{h,t} + \zeta_{h,t} )} .\]
The intertemporal budget constraint is defined as:
\[ A_{h,t+1} = (1 + r_{t+1}) \left( A_{h,t} + \left(w_{h,t}^f(H-l_{h,t}) - F(a_{h,t}) \right)P_{h,t} + y_{h,t}^m - c_{h,t} \right)\]
